\documentclass[a4paper]{book}
\usepackage[margin=1in]{geometry}
\usepackage[T1]{fontenc}
\usepackage[utf8]{inputenc}
\usepackage{hyperref}
\usepackage{amsmath}
\usepackage{amssymb}
\usepackage{enumerate}
\usepackage{setspace}
\usepackage{listings}
\usepackage{xcolor}
\usepackage{amsthm}
\usepackage{xspace}
\usepackage{adjustbox}
\usepackage{booktabs} 
\usepackage{array}    
\usepackage{colortbl}

\newtheorem*{noteenv}{Note}

% Hyperlink setup for colored links
\hypersetup{
    colorlinks = true,
    linkcolor  = blue!50!black,
    citecolor  = blue!50!black,
    urlcolor   = blue!80!black
}

% Configuration for code listings
\definecolor{codebg}{RGB}{245,245,245}
\lstset{
    backgroundcolor=\color{codebg},
    basicstyle=\ttfamily\small,
    frame=single,
    breaklines=true,
    columns=fullflexible,
    keywordstyle=\color{blue},
    commentstyle=\color{gray},
    stringstyle=\color{orange},
    showstringspaces=false
}
\setstretch{1.15}

% Title Page Info
\title{\Huge \textbf{Spanning Tree Protocol (STP) Workbook}\\
       \Large Fundamentals, Variants, Tuning, and Practical Labs}
\author{\Large Written for Networking Students and Professionals}
\date{\today}

\begin{document}

% Title Page
\begin{titlepage}
    \centering
    \vspace*{4cm}
    {\Huge \textbf{Multilayer}\par}
    \vspace{0.8cm}
    
    \rule{0.9\textwidth}{1pt}
    
    \vspace{0.6cm}
    {\large \textbf{Mehdi JAFARI ZADEH}}\par
    \vspace{0.3cm}
    
    \vfill
    \textbf{Date:} \today
    \vspace{2cm}
\end{titlepage}

% Table of Contents
\tableofcontents
\newpage

\chapter{Tutorial: Inter-VLAN Routing with ROAS and Multilayer Switches}
\section{Introduction}
Virtual Local Area Networks (VLANs) are used in network environments to segment traffic and improve performance and security. However, devices in separate VLANs cannot communicate with each other by default. Inter-VLAN routing allows devices from different VLANs to communicate, and this can be achieved through two main methods: \textbf{Router-on-a-Stick (ROAS)} and \textbf{Multilayer Switch (MLS) Inter-VLAN Routing}.

\section{Router-on-a-Stick (ROAS)}
\subsection{Overview}
Router-on-a-Stick (ROAS) is a method of inter-VLAN routing that uses a single router with a trunk link to a switch. The router performs all routing functions between VLANs.

\subsection{How It Works}
\begin{enumerate}
    \item A trunk link is established between the router and the switch.
    \item Subinterfaces are created on the router for each VLAN.
    \item Each subinterface is assigned an IP address that acts as the default gateway for its respective VLAN.
    \item The switch forwards VLAN-tagged traffic to the router, which then routes it to the appropriate VLAN.
\end{enumerate}

\subsection{Configuration Steps}
\subsubsection{On the Switch}
\begin{lstlisting}
configure terminal
interface GigabitEthernet0/1
switchport mode trunk
switchport trunk allowed vlan 10,20
exit
\end{lstlisting}

\subsubsection{On the Router}
\begin{lstlisting}
configure terminal
interface GigabitEthernet0/1
no shutdown
interface GigabitEthernet0/1.10
encapsulation dot1Q 10
ip address 192.168.10.1 255.255.255.0
exit
interface GigabitEthernet0/1.20
encapsulation dot1Q 20
ip address 192.168.20.1 255.255.255.0
exit
\end{lstlisting}

\subsection{Advantages and Disadvantages}
\textbf{Advantages:}
\begin{itemize}
    \item Cost-effective for small networks.
    \item Simple to implement.
\end{itemize}

\textbf{Disadvantages:}
\begin{itemize}
    \item Limited scalability due to reliance on a single router interface.
    \item Potential bottleneck due to the router processing all inter-VLAN traffic.
\end{itemize}

\section{Inter-VLAN Routing with a Multilayer Switch (MLS)}
\subsection{Overview}
A multilayer switch (MLS) can perform routing functions at Layer 3, eliminating the need for an external router.

\subsection{How It Works}
\begin{enumerate}
    \item The MLS is configured with Switched Virtual Interfaces (SVIs) for each VLAN.
    \item Each SVI is assigned an IP address that acts as the default gateway for its VLAN.
    \item The MLS handles inter-VLAN routing internally, reducing latency.
\end{enumerate}

\subsection{Configuration Steps}
\subsubsection{Enable Routing on the Switch}
\begin{lstlisting}
configure terminal
ip routing
\end{lstlisting}

\subsubsection{Create and Configure SVIs}
\begin{lstlisting}
interface vlan 10
ip address 192.168.10.1 255.255.255.0
no shutdown
exit
interface vlan 20
ip address 192.168.20.1 255.255.255.0
no shutdown
exit
\end{lstlisting}

\subsection{Advantages and Disadvantages}
\textbf{Advantages:}
\begin{itemize}
    \item Higher performance as traffic remains within the switch.
    \item Scalability for larger networks.
    \item Reduced latency due to elimination of external routing hops.
\end{itemize}

\textbf{Disadvantages:}
\begin{itemize}
    \item Higher cost compared to ROAS.
    \item Requires a switch with Layer 3 capabilities.
\end{itemize}

\section{Choosing Between ROAS and MLS}
\begin{table}[h]
    \centering
    \begin{tabular}{lcc}
        \toprule
        Feature & Router-on-a-Stick (ROAS) & Multilayer Switch (MLS) \\
        \midrule
        Cost & Lower & Higher \\
        Performance & Slower (bottlenecks) & Faster \\
        Scalability & Limited & Scalable \\
        Complexity & Simple & Moderate \\
        Hardware Needed & Router \& L2 Switch & L3 Switch \\
        \bottomrule
    \end{tabular}
    \caption{Comparison of ROAS and MLS}
    \label{tab:comparison}
\end{table}

For small networks with limited VLANs, ROAS is a cost-effective solution. For larger enterprise networks requiring higher performance, MLS is the preferred method.

\section{Conclusion}
Inter-VLAN routing is essential for modern networks to ensure seamless communication between VLANs. While ROAS is a simple and cost-effective solution, MLS offers improved performance and scalability. Choosing the right method depends on network size, traffic load, and budget considerations. Implementing the correct inter-VLAN routing solution ensures an efficient, high-performing, and well-segmented network infrastructure.


\part{Exercises and Hands-On Practice}

\chapter{Router-on-a-Stick (ROAS)}

\begin{enumerate}
    \item Configure the \texttt{GigabitEthernet0/0} interface as a trunk.
    \item Assign subinterface \texttt{0/0.10} for VLAN 10 with IP \texttt{192.168.10.1/24}.
    \item Assign subinterface \texttt{0/0.20} for VLAN 20 with IP \texttt{192.168.20.1/24}.
    \item Assign subinterface \texttt{0/0.30} for VLAN 30 with IP \texttt{192.168.30.1/24}.
    \item Enable 802.1Q encapsulation on subinterface \texttt{GigabitEthernet0/0.10}.
    \item Enable 802.1Q encapsulation on subinterface \texttt{GigabitEthernet0/0.20}.
    \item Enable 802.1Q encapsulation on subinterface \texttt{GigabitEthernet0/0.30}.
    \item Enable IP routing on the router.
    \item Verify VLANs on the switch.
    \item Configure \texttt{FastEthernet0/1} as a trunk on Switch1.
    \item Configure \texttt{FastEthernet0/1} as a trunk on Switch2.
    \item Configure \texttt{FastEthernet0/1} as a trunk on Switch3.
    \item Assign \texttt{FastEthernet0/2} on Switch1 to VLAN 10.
    \item Assign \texttt{FastEthernet0/2} on Switch2 to VLAN 20.
    \item Assign \texttt{FastEthernet0/2} on Switch3 to VLAN 30.
    \item Set PC1's gateway to \texttt{192.168.10.1}.
    \item Set PC2's gateway to \texttt{192.168.20.1}.
    \item Set PC3's gateway to \texttt{192.168.30.1}.
    \item Verify trunking status on switches.
    \item Show the current VLAN configuration.
    \item Display the routing table to check inter-VLAN routing.
    \item Show all subinterfaces configured on the router.
    \item Check encapsulation settings for all subinterfaces.
    \item Verify if PC1 can ping PC2.
    \item Verify if PC3 can ping PC1.
    \item Save the running configuration.
    \item Debug IP routing issues.
    \item Remove subinterface \texttt{0/0.30}.
    \item Disable the trunk link on Switch1.
    \item Reset the router configuration.
\end{enumerate}

\newpage

\chapter{Multilayer Switch (MLS)}

\begin{enumerate}
    \item Enable IP routing on the multilayer switch.
    \item Create VLAN 10 on the switch.
    \item Create VLAN 20 on the switch.
    \item Create VLAN 30 on the switch.
    \item Assign \texttt{interface VLAN 10} an IP address \texttt{192.168.10.1/24}.
    \item Assign \texttt{interface VLAN 20} an IP address \texttt{192.168.20.1/24}.
    \item Assign \texttt{interface VLAN 30} an IP address \texttt{192.168.30.1/24}.
    \item Assign \texttt{interface FastEthernet0/1} to VLAN 10.
    \item Assign \texttt{interface FastEthernet0/2} to VLAN 20.
    \item Assign \texttt{interface FastEthernet0/3} to VLAN 30.
    \item Configure trunking on \texttt{FastEthernet0/24}.
    \item Allow VLAN 10, 20, and 30 on trunk \texttt{FastEthernet0/24}.
    \item Set \texttt{FastEthernet0/24} as a dynamic desirable trunk.
    \item Assign PC1's default gateway to \texttt{192.168.10.1}.
    \item Assign PC2's default gateway to \texttt{192.168.20.1}.
    \item Assign PC3's default gateway to \texttt{192.168.30.1}.
    \item Verify VLAN assignments on the switch.
    \item Display trunk ports status.
    \item Show the switch routing table.
    \item Enable HSRP (Hot Standby Router Protocol) for VLAN 10.
    \item Enable HSRP for VLAN 20.
    \item Enable HSRP for VLAN 30.
    \item Configure SVI (Switched Virtual Interface) for VLAN 10.
    \item Configure SVI for VLAN 20.
    \item Configure SVI for VLAN 30.
    \item Show all SVI interfaces.
    \item Verify if PC1 can ping PC2.
    \item Verify if PC3 can ping PC1.
    \item Debug SVI connectivity issues.
    \item Reset the multilayer switch configuration.
\end{enumerate}

\part{Answer Key and Explanations}


\chapter{Router-on-a-Stick (ROAS) - Answers}

\begin{enumerate}
\item{Configure the \texttt{GigabitEthernet0/0} interface as a trunk}

\begin{lstlisting}
Router(config)# interface GigabitEthernet0/0
Router(config-if)# no shutdown
Router(config-if)# exit
\end{lstlisting}

\item{Assign subinterface \texttt{0/0.10} for VLAN 10 with IP \texttt{192.168.10.1/24}}

\begin{lstlisting}
Router(config)# interface GigabitEthernet0/0.10
Router(config-subif)# encapsulation dot1Q 10
Router(config-subif)# ip address 192.168.10.1 255.255.255.0
Router(config-subif)# exit
\end{lstlisting}

\item{Assign subinterface \texttt{0/0.20} for VLAN 20 with IP \texttt{192.168.20.1/24}}

\begin{lstlisting}
Router(config)# interface GigabitEthernet0/0.20
Router(config-subif)# encapsulation dot1Q 20
Router(config-subif)# ip address 192.168.20.1 255.255.255.0
Router(config-subif)# exit
\end{lstlisting}

\item{Assign subinterface \texttt{0/0.30} for VLAN 30 with IP \texttt{192.168.30.1/24}}

\begin{lstlisting}
Router(config)# interface GigabitEthernet0/0.30
Router(config-subif)# encapsulation dot1Q 30
Router(config-subif)# ip address 192.168.30.1 255.255.255.0
Router(config-subif)# exit
\end{lstlisting}

\item{Enable \texttt{802.1Q} encapsulation on subinterface \texttt{GigabitEthernet0/0.10}}

\begin{lstlisting}
Router(config)# interface GigabitEthernet0/0.10
Router(config-subif)# encapsulation dot1Q 10
Router(config-subif)# exit
\end{lstlisting}

\item{Enable \texttt{802.1Q} encapsulation on subinterface \texttt{GigabitEthernet0/0.20}}

\begin{lstlisting}
Router(config)# interface GigabitEthernet0/0.20
Router(config-subif)# encapsulation dot1Q 20
Router(config-subif)# exit
\end{lstlisting}

\item{Enable \texttt{802.1Q} encapsulation on subinterface \texttt{GigabitEthernet0/0.30}}

\begin{lstlisting}
Router(config)# interface GigabitEthernet0/0.30
Router(config-subif)# encapsulation dot1Q 30
Router(config-subif)# exit
\end{lstlisting}

\item{Enable IP routing on the router}

\begin{lstlisting}
Router(config)# ip routing
\end{lstlisting}

\item{Verify VLANs on the switch}

\begin{lstlisting}
Switch# show vlan brief
\end{lstlisting}

\item{Configure \texttt{FastEthernet0/1} as a trunk on Switch1.}

\begin{lstlisting}
Switch1(config)# interface FastEthernet0/1
Switch1(config-if)# switchport mode trunk
Switch1(config-if)# exit
\end{lstlisting}

\item{Configure \texttt{FastEthernet0/1} as a trunk on Switch2.}
\begin{lstlisting}
Switch2(config)# interface FastEthernet0/1
Switch2(config-if)# switchport mode trunk
Switch2(config-if)# exit
\end{lstlisting}

\item{Configure \texttt{FastEthernet0/1} as a trunk on Switch3.}
\begin{lstlisting}
Switch3(config)# interface FastEthernet0/1
Switch3(config-if)# switchport mode trunk
Switch3(config-if)# exit
\end{lstlisting}

\item{Assign \texttt{FastEthernet0/2} on Switch1 to VLAN 10}

\begin{lstlisting}
Switch1(config)# interface FastEthernet0/2
Switch1(config-if)# switchport mode access
Switch1(config-if)# switchport access vlan 10
Switch1(config-if)# exit
\end{lstlisting}

\item{Assign \texttt{FastEthernet0/2} on Switch2 to VLAN 20}

\begin{lstlisting}
Switch2(config)# interface FastEthernet0/2
Switch2(config-if)# switchport mode access
Switch2(config-if)# switchport access vlan 20
Switch2(config-if)# exit
\end{lstlisting}

\item{Assign \texttt{FastEthernet0/2} on Switch3 to VLAN 30}
\begin{lstlisting}

Switch3(config)# interface FastEthernet0/2
Switch3(config-if)# switchport mode access
Switch3(config-if)# switchport access vlan 30
Switch3(config-if)# exit
\end{lstlisting}

\section{PC Configuration}

\item{Set PC1's gateway to \texttt{192.168.10.1}}
\item{Set PC2's gateway to \texttt{192.168.20.1}}
\item{Set PC3's gateway to \texttt{192.168.30.1}}

\begin{lstlisting}
PC1> ip default-gateway 192.168.10.1
PC2> ip default-gateway 192.168.20.1
PC3> ip default-gateway 192.168.30.1
\end{lstlisting}

\item{Verify \texttt{trunking status} on switches.}

\begin{lstlisting}
Switch# show interfaces trunk
\end{lstlisting}

\item{Show current VLAN configuration}

\begin{lstlisting}
Switch# show vlan brief
\end{lstlisting}

\item{Display the routing table}

\begin{lstlisting}
Router# show ip route
\end{lstlisting}

\item{Show all subinterfaces}

\begin{lstlisting}
Router# show ip interface brief
\end{lstlisting}

\item{Check encapsulation settings}

\begin{lstlisting}
Router# show interfaces GigabitEthernet0/0.10
Router# show interfaces GigabitEthernet0/0.20
Router# show interfaces GigabitEthernet0/0.30
\end{lstlisting}

\item{Verify if PC1 can ping PC2.}
\item{Verify if PC3 can ping PC1.}
 
\begin{lstlisting}
PC1> ping 192.168.20.1
PC3> ping 192.168.10.1
\end{lstlisting}

\item{Save the running configuration}

\begin{lstlisting}
Router# write memory
Switch# write memory
\end{lstlisting}

\item{Debug IP routing issues}

\begin{lstlisting}
Router# debug ip routing
\end{lstlisting}

\item{Remove subinterface 0/0.30}

\begin{lstlisting}
Router(config)# no interface GigabitEthernet0/0.30
\end{lstlisting}

\item{Disable trunk link on Switch1}

\begin{lstlisting}
Switch1(config)# interface FastEthernet0/1
Switch1(config-if)# switchport mode access
Switch1(config-if)# exit
\end{lstlisting}

\item{Reset the router configuration}

\begin{lstlisting}
Router# write erase
Router# reload
\end{lstlisting}

\end{enumerate}

\newpage

\chapter{Multilayer Switch (MLS) - Answers}

\begin{enumerate}
    \item Enable IP routing on the multilayer switch.
    \begin{lstlisting}
    ip routing
    \end{lstlisting}

    \item Create VLAN 10 on the switch.
    \begin{lstlisting}
    vlan 10
    \end{lstlisting}

    \item Assign \texttt{interface VLAN 10} an IP address \texttt{192.168.10.1/24}.
    \begin{lstlisting}
    interface vlan 10
    ip address 192.168.10.1 255.255.255.0
    no shutdown
    \end{lstlisting}

    \item Assign FastEthernet0/1 to VLAN 10.
    \begin{lstlisting}
    interface FastEthernet0/1
    switchport mode access
    switchport access vlan 10
    \end{lstlisting}

    \item Configure trunking on \texttt{FastEthernet0/24}.
    \begin{lstlisting}
    interface FastEthernet0/24
    switchport mode trunk
    switchport trunk allowed vlan 10,20,30
    \end{lstlisting}

    \item Set PC1's default gateway.
    \begin{lstlisting}
    ip default-gateway 192.168.10.1
    \end{lstlisting}

    \item Show VLAN assignments.
    \begin{lstlisting}
    show vlan brief
    \end{lstlisting}

    \item Show the switch routing table.
    \begin{lstlisting}
    show ip route
    \end{lstlisting}

    \item Enable HSRP for VLAN 10.
    \begin{lstlisting}
    interface vlan 10
    standby 1 ip 192.168.10.254
    \end{lstlisting}

    \item Configure Switched Virtual Interface (SVI) for VLAN 20.
    \begin{lstlisting}
    interface vlan 20
    ip address 192.168.20.1 255.255.255.0
    no shutdown
    \end{lstlisting}

    \item Show all SVI interfaces.
    \begin{lstlisting}
    show ip interface brief
    \end{lstlisting}

    \item Verify connectivity between VLANs using ping.
    \begin{lstlisting}
    ping 192.168.20.1
    \end{lstlisting}

    \item Debug SVI connectivity issues.
    \begin{lstlisting}
    debug ip routing
    \end{lstlisting}

    \item Reset the multilayer switch configuration.
    \begin{lstlisting}
    erase startup-config
    reload
    \end{lstlisting}
\end{enumerate}
\chapter{Answers to ROAS Exercises}
\section{VLAN \& Trunking Configuration Answers}
\section{Router Subinterface Configuration Answers}
\section{Troubleshooting ROAS Answers}
\section{ACL and DHCP Configuration Answers}

\chapter{Answers to Multilayer Switch Exercises}
\section{VLAN and SVI Configuration Answers}
\section{IP Routing and EtherChannel Answers}
\section{Routing Protocol Configuration Answers}

\appendix
\chapter{Cisco IOS Command Reference}
\chapter{Troubleshooting Common VLAN and Inter-VLAN Issues}
\chapter{Additional Study Resources and Certification Guide}

\end{document}
