\documentclass[12pt]{article}
\usepackage[margin=1in]{geometry}
\usepackage[utf8]{inputenc}
\usepackage[T1]{fontenc}
\usepackage{hyperref}
\usepackage{xcolor}
\usepackage{listings}

\lstdefinestyle{CiscoCLI}{
  language=sh,
  basicstyle=\ttfamily\small,
  keywordstyle=\color{blue},
  commentstyle=\color{gray},
  showstringspaces=false
}

\begin{document}

\title{\textbf{High Availability Routing Protocols Workbook}}
\date{}
\maketitle

\noindent
Below is a consolidated workbook covering \textbf{HSRP}, \textbf{VRRP}, and \textbf{GLBP} configuration exercises. 
A single table of contents is provided for easy navigation. Each protocol has its own section with numbered exercises.

\tableofcontents
\newpage

%----------------------------------------------------------------------------------------
%	Part I: HSRP
%----------------------------------------------------------------------------------------

\section{Part I: HSRP Hands-On Configuration}

\subsection{Basic HSRP Configuration}

\subsubsection*{Exercise 1: Configure HSRP on SW1}
\textbf{Objective:} Set up HSRP on SW1 to serve as the default gateway for VLAN 10.

\textbf{Instructions:}
\begin{enumerate}
\item Access the switch CLI.
\item Enter global configuration mode.
\item Configure VLAN 10 and assign an IP address.
\item Enable HSRP on VLAN 10 with a virtual IP address.
\end{enumerate}

\textbf{Commands:}
\begin{lstlisting}[style=CiscoCLI]
switch(config)# interface vlan 10
switch(config-if)# ip address 192.168.1.1 255.255.255.0
switch(config-if)# standby 1 ip 192.168.1.254
\end{lstlisting}

\bigskip

\subsubsection*{Exercise 2: Configure HSRP on MLS1, MLS2, MLS3, MLS4}
\textbf{Objective:} Configure multiple multilayer switches (MLS) to participate in HSRP.

\textbf{Instructions:}
\begin{enumerate}
\item Access the CLI of each MLS (MLS1, MLS2, MLS3, MLS4).
\item Enter configuration mode.
\item Configure VLAN 10 with a unique IP address.
\item Enable HSRP and assign the same virtual IP.
\end{enumerate}

\textbf{Commands:}
\begin{lstlisting}[style=CiscoCLI]
switch(config)# interface vlan 10
switch(config-if)# ip address 192.168.1.X 255.255.255.0
switch(config-if)# standby 1 ip 192.168.1.254
\end{lstlisting}

\bigskip

\subsection{Priority and Preemption Configuration}

\subsubsection*{Exercise 3: Set HSRP Priority and Enable Preemption}
\textbf{Objective:} Configure a specific HSRP router to be the preferred active forwarder.

\textbf{Instructions:}
\begin{enumerate}
\item Set the HSRP priority higher than the default (100).
\item Enable preemption with a delay to avoid frequent role changes.
\end{enumerate}

\textbf{Commands:}
\begin{lstlisting}[style=CiscoCLI]
switch(config-if)# standby 1 priority 150
switch(config-if)# standby 1 preempt delay minimum 60
\end{lstlisting}

\bigskip

\subsection{HSRP Load Balancing}

\subsubsection*{Exercise 4: Configure Load Balancing using Multiple HSRP Groups}
\textbf{Objective:} Enable load balancing using multiple HSRP groups.

\textbf{Instructions:}
\begin{enumerate}
\item Configure two HSRP groups with different priorities and virtual IPs.
\end{enumerate}

\textbf{Commands:}
\begin{lstlisting}[style=CiscoCLI]
switch(config-if)# standby 1 ip 192.168.1.254
switch(config-if)# standby 1 priority 110
switch(config-if)# standby 2 ip 192.168.1.253
switch(config-if)# standby 2 priority 120
\end{lstlisting}

\bigskip

\subsection{HSRP Authentication and Security}

\subsubsection*{Exercise 5: Enable Authentication}
\textbf{Objective:} Secure HSRP communications with MD5 authentication.

\textbf{Instructions:}
\begin{enumerate}
\item Configure MD5 authentication with a secure key.
\end{enumerate}

\textbf{Commands:}
\begin{lstlisting}[style=CiscoCLI]
switch(config-if)# standby 1 authentication md5 key-string SECRET_KEY
\end{lstlisting}

\bigskip

\subsection{Tracking Mechanisms}

\subsubsection*{Exercise 6: Set Track Interface Status}
\textbf{Objective:} Implement tracking to adjust HSRP priority dynamically.

\textbf{Instructions:}
\begin{enumerate}
\item Enable tracking for an interface and set a decrement value.
\end{enumerate}

\textbf{Commands:}
\begin{lstlisting}[style=CiscoCLI]
switch(config-if)# track 1 interface FastEthernet0/1 line-protocol
switch(config-if)# standby 1 track 1 decrement 20
\end{lstlisting}

\bigskip

\subsection{Advanced HSRP Timers and Optimization}

\subsubsection*{Exercise 7: Configure HSRP Timers}
\textbf{Objective:} Optimize HSRP timers for faster failover.

\textbf{Instructions:}
\begin{enumerate}
\item Adjust the hello and hold timers.
\end{enumerate}

\textbf{Commands:}
\begin{lstlisting}[style=CiscoCLI]
switch(config-if)# standby 1 timers 3 10
\end{lstlisting}

\bigskip

\subsection{IPv6 HSRP Configuration}

\subsubsection*{Exercise 8: Enable HSRP for IPv6}
\textbf{Objective:} Implement HSRP for IPv6 networks.

\textbf{Instructions:}
\begin{enumerate}
\item Enable HSRP for IPv6 auto-configuration.
\end{enumerate}

\textbf{Commands:}
\begin{lstlisting}[style=CiscoCLI]
switch(config-if)# standby 1 ipv6 autoconfigure
\end{lstlisting}

\bigskip

\subsection{HSRP Troubleshooting and Verification}

\subsubsection*{Exercise 9: Verify HSRP Configuration}
\textbf{Objective:} Use CLI commands to verify the HSRP setup.

\textbf{Instructions:}
\begin{enumerate}
\item Display a summary of HSRP groups.
\item Inspect details of a specific HSRP group.
\item Check HSRP settings on VLAN 10.
\end{enumerate}

\textbf{Commands:}
\begin{lstlisting}[style=CiscoCLI]
switch# show standby brief
switch# show standby 1
switch# show standby interface vlan 10
\end{lstlisting}

\bigskip

\subsubsection*{Exercise 10: Debug HSRP}
\textbf{Objective:} Use debugging tools to analyze HSRP behavior.

\textbf{Instructions:}
\begin{enumerate}
\item Enable HSRP packet debugging.
\end{enumerate}

\textbf{Commands:}
\begin{lstlisting}[style=CiscoCLI]
switch# debug standby packets
\end{lstlisting}

\bigskip

\subsection{Full Topology Implementation}

\subsubsection*{Exercise 11: Configure HSRP Across the Entire Topology}
\textbf{Objective:} Implement HSRP on all switches and the router.

\textbf{On SW1:}
\begin{lstlisting}[style=CiscoCLI]
interface vlan 10
 ip address 192.168.1.1 255.255.255.0
 standby 1 ip 192.168.1.254
 standby 1 priority 150
 standby 1 preempt
\end{lstlisting}

\textbf{On MLS1, MLS2, MLS3, MLS4:}
\begin{lstlisting}[style=CiscoCLI]
interface vlan 10
 ip address 192.168.1.X 255.255.255.0
 standby 1 ip 192.168.1.254
\end{lstlisting}

\textbf{On R1:}
\begin{lstlisting}[style=CiscoCLI]
interface FastEthernet0/1
 ip address 192.168.1.254 255.255.255.0
\end{lstlisting}

\bigskip

\subsubsection*{Exercise 12: Test and Verify HSRP in the Network}
\textbf{Objective:} Perform tests to ensure HSRP is working correctly.

\textbf{Instructions:}
\begin{enumerate}
\item Verify HSRP roles using CLI commands.
\item Monitor packet forwarding.
\item Simulate a router failure and observe failover behavior.
\end{enumerate}

\textbf{Verification Commands:}
\begin{lstlisting}[style=CiscoCLI]
switch# show standby brief
switch# debug standby packets
\end{lstlisting}

\newpage

%----------------------------------------------------------------------------------------
%	Part II: VRRP
%----------------------------------------------------------------------------------------

\section{Part II: VRRP Hands-On Configuration}

\subsection{Basic VRRP Configuration}

\subsubsection*{Exercise 1: Configure VRRP on SW1}
\textbf{Objective:} Set up VRRP on SW1 to serve as the default gateway for VLAN 10.

\textbf{Instructions:}
\begin{enumerate}
\item Access the switch CLI.
\item Enter global configuration mode.
\item Configure VLAN 10 and assign an IP address.
\item Enable VRRP on VLAN 10 with a virtual IP address.
\end{enumerate}

\textbf{Commands:}
\begin{lstlisting}[style=CiscoCLI]
switch(config)# interface vlan 10
switch(config-if)# ip address 192.168.1.1 255.255.255.0
switch(config-if)# vrrp 1 ip 192.168.1.254
\end{lstlisting}

\bigskip

\subsubsection*{Exercise 2: Configure VRRP on MLS1, MLS2, MLS3, MLS4}
\textbf{Objective:} Configure multiple multilayer switches (MLS) to participate in VRRP.

\textbf{Instructions:}
\begin{enumerate}
\item Access the CLI of each MLS (MLS1, MLS2, MLS3, MLS4).
\item Enter configuration mode.
\item Configure VLAN 10 with a unique IP address.
\item Enable VRRP and assign the same virtual IP.
\end{enumerate}

\textbf{Commands:}
\begin{lstlisting}[style=CiscoCLI]
switch(config)# interface vlan 10
switch(config-if)# ip address 192.168.1.X 255.255.255.0
switch(config-if)# vrrp 1 ip 192.168.1.254
\end{lstlisting}

\bigskip

\subsection{Priority and Preemption Configuration}

\subsubsection*{Exercise 3: Set VRRP Priority and Enable Preemption}
\textbf{Objective:} Configure a specific VRRP router to be the preferred active forwarder.

\textbf{Instructions:}
\begin{enumerate}
\item Set the VRRP priority higher than the default (100).
\item Enable preemption to allow the highest priority router to take over.
\end{enumerate}

\textbf{Commands:}
\begin{lstlisting}[style=CiscoCLI]
switch(config-if)# vrrp 1 priority 150
switch(config-if)# vrrp 1 preempt
\end{lstlisting}

\bigskip

\subsection{VRRP Load Balancing}

\subsubsection*{Exercise 4: Configure Load Balancing using Multiple VRRP Groups}
\textbf{Objective:} Enable load balancing using multiple VRRP groups.

\textbf{Instructions:}
\begin{enumerate}
\item Configure two VRRP groups with different priorities and virtual IPs.
\end{enumerate}

\textbf{Commands:}
\begin{lstlisting}[style=CiscoCLI]
switch(config-if)# vrrp 1 ip 192.168.1.254
switch(config-if)# vrrp 1 priority 110
switch(config-if)# vrrp 2 ip 192.168.1.253
switch(config-if)# vrrp 2 priority 120
\end{lstlisting}

\bigskip

\subsection{VRRP Authentication and Security}

\subsubsection*{Exercise 5: Enable Authentication}
\textbf{Objective:} Secure VRRP communications with authentication.

\textbf{Instructions:}
\begin{enumerate}
\item Configure authentication with a secure key.
\end{enumerate}

\textbf{Commands:}
\begin{lstlisting}[style=CiscoCLI]
switch(config-if)# vrrp 1 authentication text SECRET_KEY
\end{lstlisting}

\bigskip

\subsection{Tracking Mechanisms}

\subsubsection*{Exercise 6: Set Track Interface Status}
\textbf{Objective:} Implement tracking to adjust VRRP priority dynamically.

\textbf{Instructions:}
\begin{enumerate}
\item Enable tracking for an interface and set a decrement value.
\end{enumerate}

\textbf{Commands:}
\begin{lstlisting}[style=CiscoCLI]
switch(config-if)# track 1 interface FastEthernet0/1 line-protocol
switch(config-if)# vrrp 1 track 1 decrement 20
\end{lstlisting}

\bigskip

\subsection{Advanced VRRP Timers and Optimization}

\subsubsection*{Exercise 7: Configure VRRP Timers}
\textbf{Objective:} Optimize VRRP timers for faster failover.

\textbf{Instructions:}
\begin{enumerate}
\item Adjust the advertisement interval.
\end{enumerate}

\textbf{Commands:}
\begin{lstlisting}[style=CiscoCLI]
switch(config-if)# vrrp 1 timers advertise 3
\end{lstlisting}

\bigskip

\subsection{IPv6 VRRP Configuration}

\subsubsection*{Exercise 8: Enable VRRP for IPv6}
\textbf{Objective:} Implement VRRP for IPv6 networks.

\textbf{Instructions:}
\begin{enumerate}
\item Enable VRRP for IPv6 with a link-local address.
\end{enumerate}

\textbf{Commands:}
\begin{lstlisting}[style=CiscoCLI]
switch(config-if)# vrrp 1 ipv6 address FE80::1
\end{lstlisting}

\bigskip

\subsection{VRRP Troubleshooting and Verification}

\subsubsection*{Exercise 9: Verify VRRP Configuration}
\textbf{Objective:} Use CLI commands to verify the VRRP setup.

\textbf{Instructions:}
\begin{enumerate}
\item Display a summary of VRRP groups.
\item Inspect details of a specific VRRP group.
\item Check VRRP settings on VLAN 10.
\end{enumerate}

\textbf{Commands:}
\begin{lstlisting}[style=CiscoCLI]
switch# show vrrp brief
switch# show vrrp 1
switch# show vrrp interface vlan 10
\end{lstlisting}

\bigskip

\subsubsection*{Exercise 10: Debug VRRP}
\textbf{Objective:} Use debugging tools to analyze VRRP behavior.

\textbf{Instructions:}
\begin{enumerate}
\item Enable VRRP packet debugging.
\end{enumerate}

\textbf{Commands:}
\begin{lstlisting}[style=CiscoCLI]
switch# debug vrrp packets
\end{lstlisting}

\bigskip

\subsection{Full Topology Implementation}

\subsubsection*{Exercise 11: Configure VRRP Across the Entire Topology}
\textbf{Objective:} Implement VRRP on all switches and the router.

\textbf{On SW1:}
\begin{lstlisting}[style=CiscoCLI]
interface vlan 10
 ip address 192.168.1.1 255.255.255.0
 vrrp 1 ip 192.168.1.254
 vrrp 1 priority 150
 vrrp 1 preempt
\end{lstlisting}

\textbf{On MLS1, MLS2, MLS3, MLS4:}
\begin{lstlisting}[style=CiscoCLI]
interface vlan 10
 ip address 192.168.1.X 255.255.255.0
 vrrp 1 ip 192.168.1.254
\end{lstlisting}

\textbf{On R1:}
\begin{lstlisting}[style=CiscoCLI]
interface FastEthernet0/1
 ip address 192.168.1.254 255.255.255.0
\end{lstlisting}

\bigskip

\subsubsection*{Exercise 12: Test and Verify VRRP in the Network}
\textbf{Objective:} Perform tests to ensure VRRP is working correctly.

\textbf{Instructions:}
\begin{enumerate}
\item Verify VRRP roles using CLI commands.
\item Monitor packet forwarding.
\item Simulate a router failure and observe failover behavior.
\end{enumerate}

\textbf{Verification Commands:}
\begin{lstlisting}[style=CiscoCLI]
switch# show vrrp brief
switch# debug vrrp packets
\end{lstlisting}

\newpage

%----------------------------------------------------------------------------------------
%	Part III: GLBP
%----------------------------------------------------------------------------------------

\section{Part III: GLBP Hands-On Configuration}

\subsection{Basic GLBP Configuration}

\subsubsection*{Exercise 1: Configure GLBP on SW1}
\textbf{Objective:} Set up GLBP on SW1 to serve as the default gateway for VLAN 10.

\textbf{Instructions:}
\begin{enumerate}
\item Access the switch CLI.
\item Enter global configuration mode.
\item Configure VLAN 10 and assign an IP address.
\item Enable GLBP on VLAN 10 with a virtual IP address.
\end{enumerate}

\textbf{Commands:}
\begin{lstlisting}[style=CiscoCLI]
switch(config)# interface vlan 10
switch(config-if)# ip address 192.168.1.1 255.255.255.0
switch(config-if)# glbp 1 ip 192.168.1.254
\end{lstlisting}

\bigskip

\subsubsection*{Exercise 2: Configure GLBP on MLS1, MLS2, MLS3, MLS4}
\textbf{Objective:} Configure multiple multilayer switches (MLS) to participate in GLBP.

\textbf{Instructions:}
\begin{enumerate}
\item Access the CLI of each MLS (MLS1, MLS2, MLS3, MLS4).
\item Enter configuration mode.
\item Configure VLAN 10 with a unique IP address.
\item Enable GLBP and assign the same virtual IP.
\end{enumerate}

\textbf{Commands:}
\begin{lstlisting}[style=CiscoCLI]
switch(config)# interface vlan 10
switch(config-if)# ip address 192.168.1.X 255.255.255.0
switch(config-if)# glbp 1 ip 192.168.1.254
\end{lstlisting}

\bigskip

\subsection{Priority and Preemption Configuration}

\subsubsection*{Exercise 3: Set GLBP Priority and Enable Preemption}
\textbf{Objective:} Configure a specific GLBP router to be the preferred active forwarder.

\textbf{Instructions:}
\begin{enumerate}
\item Set the GLBP priority higher than the default (100).
\item Enable preemption with a delay to avoid frequent role changes.
\end{enumerate}

\textbf{Commands:}
\begin{lstlisting}[style=CiscoCLI]
switch(config-if)# glbp 1 priority 150
switch(config-if)# glbp 1 preempt delay minimum 60
\end{lstlisting}

\bigskip

\subsection{GLBP Load Balancing}

\subsubsection*{Exercise 4: Configure Load Balancing}
\textbf{Objective:} Enable round-robin load balancing among GLBP routers.

\textbf{Instructions:}
\begin{enumerate}
\item Configure the GLBP group to use round-robin load balancing.
\end{enumerate}

\textbf{Commands:}
\begin{lstlisting}[style=CiscoCLI]
switch(config-if)# glbp 1 load-balancing round-robin
\end{lstlisting}

\bigskip

\subsection{GLBP Authentication and Security}

\subsubsection*{Exercise 5: Enable Authentication}
\textbf{Objective:} Secure GLBP communications with MD5 authentication.

\textbf{Instructions:}
\begin{enumerate}
\item Configure MD5 authentication with a secure key.
\end{enumerate}

\textbf{Commands:}
\begin{lstlisting}[style=CiscoCLI]
switch(config-if)# glbp 1 authentication md5 key-string SECRET_KEY
\end{lstlisting}

\bigskip

\subsection{Weighting and Tracking Mechanisms}

\subsubsection*{Exercise 6: Set Weighting and Track Interface Status}
\textbf{Objective:} Implement tracking to adjust GLBP weighting dynamically.

\textbf{Instructions:}
\begin{enumerate}
\item Configure GLBP weighting and define upper and lower limits.
\item Enable tracking for an interface and set a decrement value.
\end{enumerate}

\textbf{Commands:}
\begin{lstlisting}[style=CiscoCLI]
switch(config-if)# glbp 1 weighting 100 upper 90 lower 80
switch(config-if)# track 1 interface FastEthernet0/1 line-protocol
switch(config-if)# glbp 1 weighting track 1 decrement 20
\end{lstlisting}

\bigskip

\subsection{Advanced GLBP Timers and Optimization}

\subsubsection*{Exercise 7: Configure GLBP Timers}
\textbf{Objective:} Optimize GLBP timers for faster failover.

\textbf{Instructions:}
\begin{enumerate}
\item Adjust the hello and hold timers.
\item Configure redirect timers.
\end{enumerate}

\textbf{Commands:}
\begin{lstlisting}[style=CiscoCLI]
switch(config-if)# glbp 1 timers 3 10
switch(config-if)# glbp 1 timers redirect 600 timeout 14400
\end{lstlisting}

\bigskip

\subsection{IPv6 GLBP Configuration}

\subsubsection*{Exercise 8: Enable GLBP for IPv6}
\textbf{Objective:} Implement GLBP for IPv6 networks.

\textbf{Instructions:}
\begin{enumerate}
\item Enable GLBP for IPv6 auto-configuration.
\end{enumerate}

\textbf{Commands:}
\begin{lstlisting}[style=CiscoCLI]
switch(config-if)# glbp 1 ipv6 autoconfigure
\end{lstlisting}

\bigskip

\subsection{GLBP Troubleshooting and Verification}

\subsubsection*{Exercise 9: Verify GLBP Configuration}
\textbf{Objective:} Use CLI commands to verify the GLBP setup.

\textbf{Instructions:}
\begin{enumerate}
\item Display a summary of GLBP groups.
\item Inspect details of a specific GLBP group.
\item Check GLBP settings on VLAN 10.
\end{enumerate}

\textbf{Commands:}
\begin{lstlisting}[style=CiscoCLI]
switch# show glbp brief
switch# show glbp 1
switch# show glbp interface vlan 10
\end{lstlisting}

\bigskip

\subsubsection*{Exercise 10: Debug GLBP}
\textbf{Objective:} Use debugging tools to analyze GLBP behavior.

\textbf{Instructions:}
\begin{enumerate}
\item Enable GLBP packet debugging.
\end{enumerate}

\textbf{Commands:}
\begin{lstlisting}[style=CiscoCLI]
switch# debug glbp packets
\end{lstlisting}

\bigskip

\subsection{Full Topology Implementation}

\subsubsection*{Exercise 11: Configure GLBP Across the Entire Topology}
\textbf{Objective:} Implement GLBP on all switches and the router.

\textbf{On SW1:}
\begin{lstlisting}[style=CiscoCLI]
interface vlan 10
 ip address 192.168.1.1 255.255.255.0
 glbp 1 ip 192.168.1.254
 glbp 1 priority 150
 glbp 1 preempt
 glbp 1 load-balancing round-robin
\end{lstlisting}

\textbf{On MLS1, MLS2, MLS3, MLS4:}
\begin{lstlisting}[style=CiscoCLI]
interface vlan 10
 ip address 192.168.1.X 255.255.255.0
 glbp 1 ip 192.168.1.254
\end{lstlisting}

\textbf{On R1:}
\begin{lstlisting}[style=CiscoCLI]
interface FastEthernet0/1
 ip address 192.168.1.254 255.255.255.0
\end{lstlisting}

\bigskip

\subsubsection*{Exercise 12: Test and Verify GLBP in the Network}
\textbf{Objective:} Perform tests to ensure GLBP is working correctly.

\textbf{Instructions:}
\begin{enumerate}
\item Verify GLBP roles using CLI commands.
\item Monitor packet forwarding.
\item Simulate a router failure and observe failover behavior.
\end{enumerate}

\textbf{Verification Commands:}
\begin{lstlisting}[style=CiscoCLI]
switch# show glbp brief
switch# debug glbp packets
\end{lstlisting}

\end{document}
